\raggedright

{\Formular{\textbf{ROMAIN ROLLAND}}} {\Formular{\textit{(1866-1944, França), nascido no tempo em que só podia ter notícias de outros países por meio dos combatentes que retornavam das guerras, procurou a chave da fraternidade dos povos na herança materna: a música. Por essa chave, aliada ao fascínio pelas vidas elevadas, declarou"-se pacifista e “republicano, mais que francês”. Depois de mais de uma dezena de dramas históricos e filosóficos, escreveu a proposta de um {\slsc{Teatro do Povo}}, “máquina de guerra contra uma sociedade caduca” e convite a uma arte nova para um outro mundo. Em paralelo à escrita das vidas de Beethoven, Tolstói, Michelangelo, entre outros, dedicou"-se aos dez volumes de {\slsc{Jean"-Christophe}}, romance que expressa seu desejo de uma humanidade reconciliada e pelo qual recebeu o prêmio Nobel de Literatura em 1915. Acusado de ser antifrancês por posicionar"-se contra o nacionalismo durante a Primeira Guerra, publicou seu manifesto pacifista {\slsc{Au"-dessus de la mêlée}}. Em 1924 escreveu {\slsc{O clarão de Espinosa}}, esta vibrante homenagem ao encontro com a obra do filósofo, considerado por ele, assim como a música “que faz tocar o fundo da alma humana”, a base de sua confiança na composição harmoniosa das diferenças à qual fez apelo durante toda a vida.}}}




