\pagebreak
\thispagestyle{empty}

Ces pages sur Spinoza, qui font partie d'un chapitre de Confessions
inédites, intitulées~: \emph{Le voyage Intérieur}, n'ont jamais été
publiées que dans une lointaine revue d'Asie, en langue bengali~:
\emph{Probasi} (1926), par mon ami le professeur Kalidas Nag. Et je
veux, à ce sujet, raconter un fait émouvant, qui montre, une fois de
plus, la parenté des esprits d'Orient et d'Occident.

Quelques semaines après la publication, Kalidas Nag reçut, d'une prison
de l'Inde, une lettre censurée d'un jeune Bengali détenu politique. Le
prisonnier, qui avait lu le récit extatique de l'adolescent français
voyant filtrer au travers des barreaux de sa cage le blanc soleil de
l'Etre, s'était reconnu dans le jeune frère d'Europe. Et, de sa geôle
inconnue d'Asie, il tendait vers lui les mains, avec transports.

\begin{flushright}
\emph{Romain Rolland}
\end{flushright}

\pagebreak
\thispagestyle{empty}
\movetooddpage

J'ai toujours vécu, parallèlement, deux vies, -- l'une, celle du
personnage que les combinaisons des éléments héréditaires m'ont fait
revêtir, dans un lieu de l'espace et une heure du temps, -- l'autre,
celle de l'Etre sans visage, sans nom, sans lieu, sans siècle, qui est
la substance même et le souffle de toute vie. Mais de ces deux
consciences, distinctes et conjuguées, -- l'une épidermique et fugace,
-- l'autre, durable et profonde, -- la première a, comme il est naturel,
recouvert la seconde, pendant la plus grande part de mon enfance, de ma
jeunesse, et même de ma vie active et passionnelle. Ce n'est que par
soudaines explosions que la conscience souterraine, réussissant à forer
l'écorce des jours, jaillit comme un jet brûlant de puits artésien, --
pour quelques secondes seulement, -- de nouveau disparue et sucée par
les lèvres de la terre. Jusqu'aux temps accomplis de la maturité, où les
coups répétés des blessures de la vie élargissant les fissures de
l'écorce, la poussée de l'âme intérieure fraie à l'Etre caché son
thalweg et son lit de fleuve dans la plaine.

Avant d'en arriver à cet état de communion directe, où je suis à
présent, avec la Vie universelle, j'ai vécu séparé d'elle et proche,
l'entendant cheminer avec moi, sous le rocher, -- et soudain, de loin en
loin, aux instants que je m'y attendais le moins, vivifié par ces
irruptions de flots artésiens, qui me frappaient à la face et qui me
terrassaient.

J'ai noté trois de ces jets de l'âme, trois de ces Eclairs, qui
remplirent mes veines du feu qui fait battre le coeur de l'univers. La
trace de leur brûlure est restée aussi vive en mon vieux corps, que
l'épreuve a, depuis, roulé comme un galet, qu'à la minute lointaine où
elle s'imprimait dans la chair délicate et fiévreuse de l'adolescent.

Je ne livrerai ici que le récit du second de ces Eclairs~: les mots de
feu de Spinoza.

Entre seize et dix-huit ans.

Deux tragiques années. Insignifiantes, aux yeux qui n'en verraient que
l'étoffe, la vie familiale et scolaire d'un incertain adolescent. Mais
elles ont recélé les monstres dévorants du désespoir mortel. En ces
jours, non en d'autres, j'ai touché le fond du néant.

-- «~O aimable jeunesse~!~» me dit amèrement Spitteler en pensant à la
sienne\ldots{}

J'y reviendrai ailleurs\ldots{} Seule, tandis que je sombrais, la tempête de
Shakespeare, soulevant les couches profondes du morne océan, ramenait,
par remous, mon épave à la surface, pour la replonger dans la nuit. Je
dirai le compagnon que me fut alors Hamlet, et le commentaire, agrippé à
chaque mot comme un lierre, que je lui ai consacré\ldots{}

Mais dans l'esprit s'opérait une métamorphose. Puissante et déchirante.
Je muais, de corps et d'âme, de la voix comme de la pensée. A seize ans,
mon intelligence était encore fermée aux idées abstraites. Je
traversais, en aveugle, la classe de philosophie, au lycée Saint-Louis,
avec Evelyn et Darlu. Devant ces mots sans visage, sans couleur, sans
odeur, que les mains ne pouvaient palper, que la bouche ne pouvait
mordre, qui se refusaient à la caresse comme à la blessure des sens, ces
mots-machines de la métaphysique et des mathématiques, instruments de
génie créés par le cerveau, je restais privé de souffle et hostile\ldots{}
«~\emph{Fuori Barbari~!\ldots{}}~». -- Or, moins d'un an après, en classe de
philo que je redoublais à Louis-le-Grand, pour me préparer à Normale,
j'étais devenu le premier~; et l'excellent Monsieur Charpentier, mon
maître grand et gros, jubilait en donnant à pleine voix lecture à son
troupeau de mes dissertations~: où d'ailleurs, traîtreusement, metteur
en scène de la pensée, je faisais dialoguer Malebranche avec son
chien\ldots{} La porte était ouverte et j'enjambais le seuil du royaume de
l'Informe, -- sans doute en l'anthropomorphisant -- mais combien de
philosophes (et je dis des plus grands) ont été moins naïfs, ou plus
outrecuidants~!

Le cercle philosophique était assez étroit, dans la classe de Philo A de
Louis-le-Grand. Mais soigneusement pioché et retourné. Il restait
confiné entre les hautes haies du jardin de Descartes, Versailles de la
pensée. J'ai été substantifiquement nourri de la moelle Cartésienne,
pendant deux à trois années.

J'y ajoutais ce que j'allais grappiller dans l'enclos voisin (Philo B),
à la vigne de Burdeau, des fantasmagories Présocratiques. Quelques
graines tombées du bec de ces grands oiseaux, Ioniens et Trinacriens,
ont germé depuis, dans mon «~\emph{Empédocle d'Agrigente}~».

-- Mais le chemin naturel de l'esprit me conduisait à ceux qui, partis
du majestueux jardin muré de Descartes, y avaient, par une brèche,
ouvert des perspectives illimitées. Il me mena tout droit, d'instinct,
comme un chien, guidé par le flair de deux ou trois mots -- à Spinoza.

J'ai gardé précieusement l'édition, devenue rare, achetée sous les
galeries de l'Odéon, -- qui fut, en ces années, mon élixir de vie
éternelle~:

\emph{Oeuvres de Spinoza, traduites par Emile Saisset, avec une
introduction critique, nouvelle édition revue et augmentée}.
Charpentier, 1872, 3 volumes, in-12, cartonnés en vert.

Bien que ma pensée soit maintenant affranchie du strict rationalisme de
maître Benoît, et qu'elle en ait reconnu maints paralogismes, il me
reste sacré, à l'égal des Livres Saints pour un qui croit en eux~; et je
ne touche ces trois volumes qu'avec un pieux amour.

Je n'oublierai jamais que dans le cyclone de mon adolescence, j'ai
trouvé mon refuge au nid profond de l'\emph{Ethique}.

C'est quatre heures. L'hiver. Le jour tombe. Le jour terne d'un ciel
gris et glacé. Je suis assis devant ma table adossée au mur, près de la
fenêtre. Dehors, la rue Michelet, déserte, où s'engouffre la bise, et,
séparé par une grille, le funèbre jardin de l'Ecole de pharmacie, où les
rares visiteurs semblent prier devant les tombes des plantes\footnote{Depuis,
  la verdure a grandi. Alors, le jardin venait de s'ouvrir~; il était
  tout pierreux.}. Mais je ne vois rien du dehors. Je suis muré. Muré
dans la chambre close. Muré dans ma carapace hérissée contre le froid,
qui pénètre dans la pièce non chauffée et jusque sous le pardessus où se
recroqueville mon corps frileux. Muré dans la contemplation du livre,
que tiennent mes doigts gourds. Autour de moi, je sens, morne halo, le
triste jour qui meurt, l'implacable nature, l'étau de la ville de
pierre, et celui de mes pensées.

L'éternel prisonnier, attaché dans sa geôle, traîne au pied le boulet du
souci, de la lutte pour la vie, l'obsession acharnée de l'examen, qui
empoisonne tant de jeunes existences, des échecs répétés, de la
nécessité de crisper toutes ses forces pour le combat, l'obligation
morale de vaincre, non seulement pour vivre, pour sauver sa vie, mais
pour sauver celle des siens, pour répondre à leur sacrifice absolu, qui
a misé tout leur sort sur une carte, sur mon sort. Malheureux enfant
débile, sur qui pèse une responsabilité trop lourde, qu'il n'a pas
demandée~! Elle l'oppresse~; et pourtant, elle lui est une armure~; en
écrasant ses épaules, elle l'oblige à se raidir. Sans elle, il
s'abandonnerait au Rêve incessant, qui bourdonne dans le fond de la
ruche fermée. Mais sous la chape qui le recouvre, sa frêle et nerveuse
énergie se concentre, se tend, angoissée, vers une lueur qui filtre par
l'étroit soupirail\ldots{}

Elle filtre. Je la fixe dans la nuit de ma cave. Je la fixe entre les
barreaux noirs des lignes du livre vêtu de vert. Et sous la fixité
trouble de mon regard halluciné, voici que les barreaux s'écartent et
que surgit le soleil blanc de la \emph{Substance}. Métal en fusion, qui
remplit la coupe de mes yeux, il coule dans mon être qu'il consume~; et
mon être, comme une fonte, rejaillit en la cuve\ldots{}

Il a suffit d'une page, la première, -- de quatre Définitions, et de
quelques éclats de feu qui ont sauté, au choc des silex de
l'\emph{Ethique}\footnote{\emph{Éthique}, \versal{I}. \emph{Définitions} 3, 4, 5,
  6 et l'\emph{Explication}, qui suit. Etincelles arrachées aux
  propositions 15 et 16 de \versal{I}, et à la \emph{Scholie} du \emph{Lemme} 7
  de \versal{II}.}.

Je ne me fais point d'illusion, et ne veux pas en faire aux autres. Je
ne prétends, ni que cette vertu de miracle soit inhérente à des mots
magiques, ni que j'y aie alors saisi la pensée vraie de Spinoza. De même
qu'en lisant le long premier volume d'\emph{Introduction}, honnête et
timorée, par Emile Saisset, je ne m'arrêtai pas aux arguments
effarouchés de ce spiritualiste et sautai allègrement pardessus son
garde-feu dans le braisier, dont son labeur avait pour objet de me
défendre -- (Naïfs contradicteurs~! C'est à eux que l'ont doit de
connaître et d'aimer les génies interdits~!) -- ainsi, dans le texte
même de Spinoza, je découvrais non lui, mais moi ignoré. Dans
l'inscription tracée au porche de l'\emph{Ethique}, dans ces Définitions
aux lettres flamboyantes, je déchiffrais, non ce qu'il avait dit, mais
ce que je voulais dire, les mots que ma propre pensée d'enfant, de sa
langue inarticulée, s'évertuait à épeler. On ne lit jamais un livre. On
se lit à travers les livres, soit pour se découvrir, soit pour se
contrôler. Et les plus objectifs sont les plus illusionnés. Le plus
grand livre n'est pas celui dont le communiqué s'imprimerait au cerveau,
ainsi que sur le rouleau de papier un message télégraphique, mais celui
dont le choc vital éveille d'autres vies, et, de l'une à l'autre,
propage son feu qui s'alimente des essences diverses et, devenu
incendie, de forêt en forêt bondit.

Je n'essaierai donc pas d'expliquer ici le sens libérateur de la vraie
pensée de Spinoza, mais celui que j'y ai trouvé, parce que depuis
l'enfance mon obscure passion, à tatons, le cherchait.

Et certes, ce n'est point le maître de l'ordre géométrique --
«~\emph{Ethica ordine geometrico demonstrata}~» -- ce n'est point le
rationaliste qui m'a conquis en Spinoza, -- quelque jouissance
esthétique que me procurent les jeux magnifiques de la raison~: -- c'est
le réaliste.

Qu'il est étrange que \sout{c'est} cet aspect de la grande figure soit
recouvert, jusqu'à devenir invisible, par le lourd verbalisme
intellectuel des philosophes de profession~! Comment, du premier regard,
ne saisissent-ils pas ce regard, cette voix, ivres du Réel~!

\begin{quote}
Il est absolument nécessaire de tirer toutes nos idées des choses
physiques, c'est à dire des êtres réels, en allant, suivant la série des
causes, d'un être réel à un autre être réel, sans passer aux choses
abstraites et universelles, ni pour en conclure rien de réel, ni pour
les conclure de quelque être réel~: car l'un et l'autre interrompent la
marche véritable de l'entendement.
\end{quote}

N'est-ce pas un principe de réalisme halluciné, qui gouverne en ces mots
le \emph{Traité de l'Entendement}\footnote{Edition Saisset, tome \versal{III}, p.
  338.}, ajoutant aussitôt après, avec l'imperturbable assurance du
visionnaire~:

\emph{\ldots{} Mais il faut remarquer que par la série des causes et des
êtres réels, je n'entends point ici la série des choses particulières et
changeantes, mais seulement la série des choses fixes et
éternelles.}\footnote{«\emph{Per realitatem et perfectionem idem
  intelligo. »} (\versal{II}. Déf. 6).}

«~\emph{Les choses fixes et éternelles~}» sont «~\emph{réelles}~». Elles
sont \emph{le plus réel}. Et tout ce qui est \emph{réel} est
\emph{individuel}.

«~\emph{Les choses fixes et éternelles}~» sont
«~\emph{particulières}~».\footnote{P. 329} Point d'abstractions. Des
Essences. Des Etres. Tout est \emph{être~}: -- et les \emph{Modes}
innombrables et finis~; et l'infinité des \emph{Attributs} infinis~; et
l'Etre des êtres, la Substance, «~\emph{L'Etre unique, infini, l'être
qui est tout l'être, et hors duquel il n'y a rien}~»\footnote{P. 329}.

Vertige~!\ldots{} Vin de feu~!\ldots{} Ma prison s'ouvre. Voilà donc la réponse,
obscurément conçue dans la douleur et dans le désespoir, appelée par des
cris de passion aux ailes brisées, obstinément cherchée, voulue, dans
les meurtrissures et les larmes de sang, la voilà rayonnante, la réponse
à l'énigme du Sphinx, qui m'étreint depuis l'enfance, -- à l'antinomie
accablante entre l'immensité de mon être intérieur et le cachot de mon
individu, qui m'humilie et qui m'étouffe~!

«~\emph{Nature naturante}~» et «~\emph{nature naturée}~»\footnote{\emph{Ethique},
  \versal{I} \emph{Scholie} à 29.}\ldots{}

C'est la même.

«~\emph{Tout ce qui est, est en Dieu}.~»\footnote{Ibid., \versal{I}, 15.}

Et moi aussi, je suis en Dieu~! De ma chambre glacée, où tombe la nuit
d'hiver, je m'évade au gouffre de la Substance, dans le soleil blanc de
l'Etre.

Horizons inouïs~! Mon rêve, même en ses vols les plus délirants, est
dépassé. Non seulement mon corps et mon esprit, mon univers, baignent
dans des mers sans rivages, l'Etendue, la Pensée, dont nulle caravelle
ne pourra faire le tour. Mais, dans l'insondable immensité, j'entends
bruire, à l'infini, d'autres mers, d'autres mers inconnues, des
Attributs innommables, inconcevables, à l'infini. Et tous sont contenus
en l'Océan de l'Etre. Entre son pouce et son petit doigt, ils tiennent à
l'aise. L'intuition de Spinoza ouvre les cieux fermés, -- de deux
siècles en avance, pionnière des \emph{conquistadores} de la science
moderne. Et si, dans ces Nouveaux Mondes, elle sait et nous dit que,
sous notre forme humaine, nous n'aborderons jamais, elle nous communique
l'ivresse de la certitude qu'ils existent, qu'ils sont là, près de
nous~: ce n'est pas seulement un fait de connaissance, mais le battement
de coeur d'une coexistence. Enrichissement prodigieux de mon univers, il
n'y a qu'un instant étranglé dans la cage de ma maigre poitrine~! Et mon
coeur ne souffre pas de son énormité. Les ailes étendues, planant sur
ces espaces, souffle à souffle, seul à seul, fixant le regard, sans
ciller de la Face omniprésente -- «~ \emph{Facies totius
universi}~»\footnote{Lettre \versal{XLIV} à Schuller.} -- je me sens soutenu par
l'infaillible main de la Libre Nécessité, qui émane du Dieu. Je ne
tomberai point. Car je suis sien. Ma chute serait la sienne\ldots{}

\emph{Si una pars materiae annihilaretur, simul tota Extensio
evanesceret\ldots{}}\footnote{Lettre \sout{\versal{XL}}\versal{IV} à Oldenburg.}

Je ne puis tomber qu'en Lui. Je suis calme.

Tout est calme. Je jouis de ma plénitude et de mon harmonie\ldots{}

\emph{\ldots{} Possédant par une sorte de nécessité éternelle la connaissance
de moi-même et de Dieu et des choses, jamais je ne cesse d'être~; et la
véritable paix de l'âme, je la possède pour toujours.}

Mais ces dernières lignes de l'\emph{Ethique}, il ne faut pas les lire
-- et je ne les lisais pas -- avec les yeux froids de l'intelligence. Il
faut y apporter la passion de son coeur et l'ardeur de ses sens. Il faut
participer au spasme de cette «~Béatitude~», ainsi que lui-même il la
nomme, notre Krishna d'Europe, et qui est «~\emph{un amour}~»\footnote{«\emph{L'amour
  divin ou la béatitude\ldots{}~»}} et une volupté, -- la plus voluptueuse
des jouissances humaines~:

\emph{Aeternitatem, hoc est, infinitam existendi, sive, invita
latinitate, essendi fruitionem.}\footnote{Lettre \versal{XII} à L. Mayer.}

Goûtez la saveur sensuelle de ce latin barbare~; «~\emph{essendi
fruitio}~»~!\ldots{} De mes yeux, de mes mains, de ma langue, de tous les
pores de ma pensée, je l'ai goûtée. J'ai étreint l'Etre.

O rire de Zarathustra~! Je n'ai pas attendu Nietzsche pour te connaître.
Tu résonnes ici, mais de quelles harmonies plus belles et plus pleines~!
Et comme elles sont proches de celles de l'\emph{Ode à la Joie~}!\ldots{}

\begin{quote}
La joie est une passion qui augmente ou favorise la puissance du
corps\ldots{} La joie est bonne\ldots{} La gaieté ne peut avoir d'excès, et elle
est toujours bonne\ldots{} Le rire est un pur sentiment de joie, et il ne
peut avoir d'excès, et il est bon\ldots{} Plus nous avons de joie, et plus
nous avons de perfection\ldots{}\footnote{\emph{Ethique}, \versal{IV}, 41, 42, 45,
  \emph{Scholie}.}

\ldots{} Jouir de la nourriture, des parfums, des couleurs, des beaux
vêtements, de la musique, des jeux, des spectacles, et de tous les
divertissements que chacun peut se donner, sans dommage pour
personne\ldots{}

\ldots{} User des choses de la vie et en jouir autant que possible\ldots{}
Se réunir aux autres et tâcher de les unir, -- car tout ce qui
tend à les unir est bon -- s'efforcer de partager sa joie avec les
autres\ldots{}\footnote{\emph{Ethique}, \versal{IV}, 40.} \emph{-- s'unir, en pleine
connaissance, avec toute la nature\ldots{}}\footnote{\emph{Réforme de
  l'Entendement.}}

\emph{Seid umschlungen, Millionen~!\ldots{}}
\end{quote}

Embrassons-nous, millions d'êtres~!

\begin{flushright}
Juillet 1924
\end{flushright}