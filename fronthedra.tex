\AtBeginDocument{%
\begingroup\pagestyle{empty}\raggedright\parindent0pt
%\vspace*{-90pt}
%\begin{changemargin}{-60pt}{-60pt}
%{\fontsize{70}{1000}\Cobraarisca{
%\mbox{\noindent\rlap{\enspace C C C C C}{C C C C C C}}
%\mbox{\noindent\rlap{\enspace C C C C C}{C C C C C C}}
%\mbox{\noindent\rlap{\enspace C C C C C}{C C C C C C}}
%\mbox{\noindent\rlap{\enspace C C C C C}{C C C C C C}}
%\vspace{-18pt}{
%\mbox{\noindent\rlap{\enspace C C C C C}{C C C C C C}}}
%\mbox{\noindent\rlap{\enspace C C C l C}{C C C C C C}}
%\mbox{\noindent\rlap{\enspace C C C C C}{C C C C C C}}
%\mbox{\noindent\rlap{\enspace C C C C C}{C C C C C C}} \enlargethispage{\textheight}
%\mbox{\noindent\rlap{\enspace C C C C C}{C C C C C C}}
%\mbox{\noindent\rlap{\enspace C C C C C}{C C C C C C}}
%\mbox{\noindent\rlap{\enspace C C C C C}{C C C C C C}}
%}}
%\end{changemargin}
\begin{flushright}
\begin{vplace}[30]
``(\ldots{}) Um homem toma posse de si mesmo por meio de lampejos, e muitas vezes quando toma posse de si não se encontra nem se alcança. (\ldots{})''

\vspace*{0.2cm}

A. Artaud,
\textit{Carta para Jacques Rivière} em 25 de maio de 1924
\end{vplace}
\end{flushright}

%\LARGE{\autor}
%\vfill
\clearpage

%% Créditos ------------------------------------------------------
\raggedright

%{\Formular{\normalsize{N-1 edições \hspace{4.6cm} Coleção
%
%\vspace*{-5mm}
%
%\& Hedra \hspace{5cm} Lampejos}}}

%\bigskip
\linhalayout{Coleção Lampejos}{}
{\tiny{©n-1 edições 2020 / Hedra}}

\vspace*{0.5cm}

\linhalayout{\emph{O clarão de Espinosa}}{}
\linhalayout{\emph{Romain Rolland}}{}
\linha{título original}{\titulooriginal}
{\tiny{©n-1 edições 2020}}

\vspace*{1cm}

\linha{tradução©}{\copyrighttraducao}
\linhalayout{coordenação editorial}{Peter Pál Pelbart e Ricardo Muniz Fernandes}
\linhalayout{direção de arte}{Ricardo Muniz Fernandes}
%\linhalayout{revisão}{}
\linhalayout{projeto da coleção/capa}{Lucas Kröeff}
\linhalayout{ilustração/alfabeto}{Waldomiro Mugrelise}
\linha{coedição}{\coedicao}
\linha{assistência editorial}{\assistencia}
\linha{ISBN}{\ISBN}\smallskip
\linha{revisão}{\revisao}
\linha{preparação}{\preparacao}
\linha{iconografia}{\iconografia}
\linha{imagem da capa}{\imagemcapa}\smallskip
\begingroup\tiny
%\ifdef{\conselho}{\conselho}{\relax}
\par\endgroup\bigskip
%
\begingroup \tiny
%
\textit{Grafia atualizada segundo o Acordo Ortográfico da Língua\\
Portuguesa de 1990, em vigor no Brasil desde 2009.}\\
%
\begin{vplace}[30]
\vfill\textit{Direitos reservados em l\'ingua\\ portuguesa somente para o Brasil}\\\medskip
%%
\textsc{n-1 edições}\\
R.~Frei Caneca, 322 (cj 52)\\
São Paulo-\textsc{sp}, Brasil\\
\end{vplace}

\endgroup

%{\scriptsize\fakereceipt{
%A CONTRACULTURA,\\
%ENTRE A CURTIÇÃO\\
%E O EXPERIMENTAL

%\vspace{1cm}

%CELSO FAVARETTO

%\vspace{2cm}

%DESENHOS\\
%\vspace{0.5cm}
%WALDOMIRO\\
%\hspace{1.7cm}MUGRELISE\\

%\vspace{1cm}

%COLEÇÃO LAMPEJOS

%\vspace{0.5cm}

%PRIMEIRA EDIÇÃO

%\vspace{0.5cm}

%COPYRIGHT: HEDRA \& N-1\\
%EDIÇÃO: RICARDO MUNIZ\\
%PROJETO DA COLEÇÃO: L.KRÖEFF\\
%TIRAGEM: 1\textsuperscript{A TIRAGEM} 300UN\\
%ISBN: XXXXXXXXXXXXXXXXXXXXX\\

%\vspace{1cm}

%PUBLICADO POR\\
%HEDRA \& N-1\\
%SÃO PAULO, 2019\\
%IMPRESSA SOB DEMANDA\\

%\vspace{0.5cm}

%RUA FRADIQUE COUTINHO, 1139\\
%05416-011 SÃO PAULO, SP BRASIL
%}}
\pagebreak

%\vspace*{-80pt}
\begin{changemargin}{-85pt}{-60pt}
{\Formular{

\mbox{\textbf{Romain Rolland} \hspace{0.6cm}\textbf{Romain Rolland} \hspace{0.6cm}\textbf{Romain Rolland} \hspace{0.6cm}\textbf{Romain Rolland}}\vspace*{-3pt}
\mbox{O clarão de \hspace{1.6cm}O clarão de \hspace{1.6cm}O clarão de \hspace{1.6cm}O clarão de}\vspace*{-3pt}
\mbox{Espinosa \hspace{2.08cm}Espinosa \hspace{2.08cm}Espinosa \hspace{2.08cm}Espinosa}\vspace*{-3pt}
}}
\end{changemargin}
%% Front ---------------------------------------------------------
% Titulo


%{\LARGE{\autor} \par}%\vspace{1.5ex}}
%\vspace{6cm} %9.3cm
%\ifdef{\organizador}{{\small {\organizador} (\textit{organização})} \par}{}
%\ifdef{\introdutor}{{\small {\introdutor} (\textit{prefácio})} \par}{}
%\ifdef{\tradutor}{{\small {\tradutor} (\textit{tradução})}\par}{}\vspace{8.5mm}

%{{\footnotesize{} \ifdef{\numeroedicao}{\numeroedicao}{2}ª edição} \par}
%logos
\vfill
%\normalsize
%\ifdef{\logo}{\IfFileExists{\logo}{\hfill\includegraphics[width=3cm]{\logo}\hfill\logoum{}\\ São Paulo\_\the\year}}{\logoum\break{} São Paulo\_\the\year}
%\includegraphics[width=.4\textwidth,trim=0 0 25 0]{logo.jpg}\\\smallskip
\par\clearpage\endgroup
% Resumo -------------------------------------------------------
\begingroup \footnotesize \parindent0pt \parskip 5pt \thispagestyle{empty} \vspace{1cm}%\textheight}\mbox{} \vfill
\movetoevenpage
\raggedright
{\Formular{\textit{O livro como imagem do mundo é de toda maneira uma ideia insípida. Na verdade não basta dizer Viva o múltiplo, grito de resto difícil de emitir. Nenhuma habilidade tipográfica, lexical ou mesmo sintática será suficiente para fazê-lo ouvir. É preciso fazer o múltiplo, não acrescentando sempre uma dimensão superior, mas, ao contrário, da maneira mais simples, com força de sobriedade, no nível das dimensões de que se dispõe, sempre n-1 (é somente assim que o uno faz parte do múltiplo, estando sempre subtraído dele). Subtrair o único da multiplicidade a ser constituída; escrever a n-1.}}}

\vspace*{4pt}

{\Formular{\textit{Gilles Deleuze e Félix Guattari}}}
\movetooddpage
\baselineskip=.92\baselineskip
\IfFileExists{PRETAS.tex}{\raggedright

{\Formular{\textbf{ROMAIN ROLLAND}}} {\Formular{\textit{\lipsum[1]}}}




}{% 
\ifdef{\resumo}{\resumo\par}{}
\ifdef{\sobreobra}{\sobreobra}{}
\ifdef{\sobreautor}{\mbox{}\vspace{4pt}\newline\sobreautor}{}
\ifdef{\sobretradutor}{\newline\sobretradutor}{\relax}
\ifdef{\sobreorganizador}{\vspace{4pt}\newline\sobreorganizador}{\relax}\par}
\thispagestyle{empty} \endgroup
\ifdefvoid{\sobreautor}{}{\pagebreak\ifodd\thepage\paginabranca\fi}
% Sumário -------------------------------------------------------
\pagebreak
\thispagestyle{empty}
\begin{vplace}[0.08]


{\large\Formular{
\noindent{}L'éclair de Spinoza
}}

\end{vplace}

\pagebreak
\thispagestyle{empty}
\begin{vplace}[0.25]


{\large\Formular{
\noindent{}O clarão de Espinosa
}}

\vspace{8cm}
{\Formular{\noindent{}Carla Ferro\\ ({\slsc{tradução}})
}}
\end{vplace}


%\sumario{}
%\IfFileExists{INTRO.tex}{\include{INTRO}}

%\IfFileExists{TEXTO.tex}{\mbox{}\include{TEXTO}}
%\part[{{\def\break{}\titulo}}]{\titulo}
} % fim do AtBeginDocument

% Finais -------------------------------------------------------
%\AtEndDocument{%
  %\publicidade
%
%\pagebreak\ifodd\thepage\paginabranca\fi
%
%\ifdef{\imagemficha}{\IfFileExists{\imagemficha}{\includegraphics[width=.7\textwidth]{\imagemficha%}\par}}{}
%
%\mbox{}\vfill\small\thispagestyle{empty}
%\begin{center}
%\begin{minipage}{.8\textwidth}
%\centering\tiny\noindent{}Adverte-se aos curiosos que se imprimiu este livro \ifdef{\grafica}{na %gráfica \grafica}{em nossas oficinas}, 
%em \today \ifdef{\papelmiolo}{em papel \papelmiolo}, em tipologia \tipopadrao{}, com diversos %sofwares livres, 
%entre eles, Lua\LaTeX, git \& ruby. \ifdef{\RevisionInfo{}}{\par(v.\,\RevisionInfo)}{}\par \begin{%center}\normalsize\adforn{64}\end{center}
%\end{minipage}
%\end{center}
%}
