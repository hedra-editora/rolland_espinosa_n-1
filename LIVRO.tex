\documentclass[semcabeco,showtrims,trimframe,12pt,conselho,spreadimages]{memoir}

\usepackage[11x18]{hedraoptions} %% << %%%%%%%%%%%%%%%%
\usepackage[baruch]{hedrastyles}
\usepackage[xetex,chicagofootnotes]{tipografia}
\usepackage[standart,sempontinhos]{toc}
\usepackage{hedraextra}
\usepackage{penalidades}
\usepackage{graficos}
\usepackage{hedralogo}
\usepackage{hifensextras}
\usepackage{fichatecnica}
\usepackage[standart]{aparatos}
\usepackage{tabelas}
\usepackage{versos}
\usepackage{gitrevisioninfo}
\usepackage{parallel}

\newcommand{\forceindent}{\leavevmode{\parindent=1,4em\indent}}

\newcommand{\quebra}{\vfil\pagebreak}

\linespread{1.15}

\usepackage{endnotes}
\renewcommand{\notesname}{Notas}

%\counterwithin*{endnote}{part}
%\counterwithin*{endnote}{chapter}

\let\latexchapter\chapter
\makeatletter
\renewcommand\enoteheading{%
  \setcounter{secnumdepth}{-2}
  \latexchapter*{\notesname\markboth{NOTAS}{}}
  \mbox{}\par\vskip-\baselineskip
  \let\@afterindentfalse\@afterindenttrue
}
\makeatother
%\usepackage{fancyhdr}
%\pagestyle{fancy}
%\setlength{\headheight}{9mm}
%\fancyhf{}
%\fancyhead[R]{\thepage}
%\renewcommand{\headrulewidth}{0pt}

%\lhead[\fancyplain{}]{}
%\chead[\fancyplain{}]{}
%\rhead[\fancyplain{}]{\cnvt{\thepage} -- \thepage}

%\newcommand*{\cnvt}[1]{\the\numexpr#1-1\relax}

%\fancypagestyle{chapter}{
%\pagestyle{fancy}
%\setlength{\headheight}{5mm}
%\fancyhf{}
%\fancyhead[R]{\thepage}
%\renewcommand{\headrulewidth}{0pt}}

%--------------------------------------------PERMITIR FONTES MAIORES (HUGE)

\usepackage{anyfontsize}

%--------------------------------------------TIRAR TRAÇO DIVISOR DA FOOTNOTE

\usepackage{footmisc}

\renewcommand*\footnoterule{}
%\fancyhf[RO]{\cnvt{\thepage} -- \thepage}
%\fancyfoot{}
%\renewcommand{\headrulewidth}{0pt}
%\renewcommand{\footrulewidth}{0pt}}

%--------------------------------------------DEFININDO FONTES A SEREM USADAS NO LIVRO

\usepackage{fontspec}

\newcommand{\slsc}[1]{\fontspec[SmallCapsFeatures={FakeSlant=0.6}]{Formular-LightItalic}\textsc{#1}\fontspec[]{FormularLight-Italic}}

%\usepackage{Formular}
\newfontfamily\Formular{Formular-Regular}[
BoldFont = Formular-Bold.otf,
ItalicFont = Formular-Light.otf]
%\newfontfamily\Cobraarisca{Cobra arisca-Regular}
%\newfontfamily\fakereceipt{fake receipt}

%--------------------------------------------ALTERAR FONTE DA NOTA DE RODAPÉ

\setsansfont{Formular-Light}


\usepackage{etoolbox}
\makeatletter
\patchcmd{\@footnotetext}{\footnotesize}{\scriptsize\sffamily}{}{}
\makeatother

%--------------------------------------------ALTERAR FONTE DA NUMERAÇÃO DE PÁGINA
\usepackage{graphicx}
\usepackage{fancyhdr}
\pagestyle{fancy}
\fancyhf{}
\fancyfoot[CE,CO]{\Formular \footnotesize \textit \thepage}
\renewcommand{\headrulewidth}{0pt}

%--------------------------------------------ALTERAR DISTÃNCIA ENTRE TÍTULO DO SUMÁRIO E CAPÍTULOS
%\addtocontents{toc}{\vskip-15pt}
%--------------------------------------------
\usepackage{afterpage}

\newcommand\blankpage{%
    \null
    \thispagestyle{empty}%
    \addtocounter{page}{0}%
    \newpage}


\newenvironment{changemargin}[2]{%
\begin{list}{}{%
\setlength{\topsep}{0pt}%
\setlength{\leftmargin}{#1}%
\setlength{\rightmargin}{#2}%
\setlength{\listparindent}{\parindent}%
\setlength{\itemindent}{\parindent}%
\setlength{\parsep}{\parskip}%
}%
\item[]}{\end{list}}



\usepackage{changepage}
\usepackage{placeins}
\usepackage{float}
\usepackage{floatpag}
\usepackage{rotating}
%\usepackage{afterpage}
\usepackage{paracol}
\setlength{\columnsep}{0.8cm}

\newenvironment{absolutelynopagebreak}
  {\par\nobreak\vfil\penalty0\vfilneg
   \vtop\bgroup}
  {\par\xdef\tpd{\the\prevdepth}\egroup
   \prevdepth=\tpd}

%\usepackage{imakeidx} 
%\makeindex[program=xindy, options=-C utf8 -L portuguese]
%\newcommand\gobbleone[1]{}
%\newcommand*{\seeonly}[2]{\ (\emph{\seename} #1)}
%\newcommand*{\also}[2]{\emph{cf.} #1}
%\newcommand{\Also}[2]{\emph{See also} #1}
%\renewcommand\indexname{Índice onomástico}
%\makeindex[intoc]

\setcounter{tocdepth}{1}
\setcounter{secnumdepth}{-2} 
%\linespread{1.08}

\makeatletter
\newenvironment{Parskip}{%
   \par
   \parskip=0.3\baselineskip \advance\parskip by 0pt plus 2pt
   \parindent=\z@
   \def\@listI{\leftmargin\leftmargini
      \topsep\z@ \parsep\parskip \itemsep\z@}
   \let\@listi\@listI
   \@listi
   \def\@listii{\leftmargin\leftmarginii
      \labelwidth\leftmarginii\advance\labelwidth-\labelsep
      \topsep\z@ \parsep\parskip \itemsep\z@}
   \def\@listiii{\leftmargin\leftmarginiii
       \labelwidth\leftmarginiii\advance\labelwidth-\labelsep
       \topsep\z@ \parsep\parskip \itemsep\z@}
   \partopsep=\z@
}{\par}
\makeatother

\makeatletter
\newenvironment{myParskip}{%
   \par
   \parskip=0.2\baselineskip \advance\parskip by 0pt plus 2pt
   \parindent=\z@
   \def\@listI{\leftmargin\leftmargini
      \topsep\z@ \parsep\parskip \itemsep\z@}
   \let\@listi\@listI
   \@listi
   \def\@listii{\leftmargin\leftmarginii
      \labelwidth\leftmarginii\advance\labelwidth-\labelsep
      \topsep\z@ \parsep\parskip \itemsep\z@}
   \def\@listiii{\leftmargin\leftmarginiii
       \labelwidth\leftmarginiii\advance\labelwidth-\labelsep
       \topsep\z@ \parsep\parskip \itemsep\z@}
   \partopsep=\z@
}{\par}
\makeatother

\newcommand{\mystar}{{\fontfamily{lmr}\selectfont$\star$}}

%\makeatletter
%\renewcommand{\@chapapp}{}% Not necessary...
%\newenvironment{chapquote}[2][2em]
%  {\setlength{\@tempdima}{#1}%
%   \def\chapquote@author{#2}%
%   \parshape 1 \@tempdima \dimexpr\textwidth-2\@tempdima\relax%
%   \itshape}
%  {\par\scriptsize\hfill-- \chapquote@author\hspace*{\@tempdima}\par\bigskip}
%\makeatother

%\newcommand\Chapter[2]{\chapter
%  [#1\hfil\hbox{}\protect\linebreak{\itshape#1}]%
%  {#1\\[2ex]\Large\itshape#2}%
%}

\begin{document}

\begin{Parallel}[p]{}{} 
\ParallelLText{\selectlanguage{french} %\pagebreak
%\thispagestyle{empty}


Ces pages sur Spinoza, qui font partie d'un chapitre de Confessions
inédites, intitulées~: \emph{Le voyage Intérieur}, n'ont jamais été
publiées que dans une lointaine revue d'Asie, en langue bengali~:
\emph{Probasi} (1926), par mon ami le professeur Kalidas Nag. Et je
veux, à ce sujet, raconter un fait émouvant, qui montre, une fois de
plus, la parenté des esprits d'Orient et d'Occident.

Quelques semaines après la publication, Kalidas Nag reçut, d'une prison
de l'Inde, une lettre censurée d'un jeune Bengali détenu politique. Le
prisonnier, qui avait lu le récit extatique de l'adolescent français
voyant filtrer au travers des barreaux de sa cage le blanc soleil de
l'Etre, s'était reconnu dans le jeune frère d'Europe. Et, de sa geôle
inconnue d'Asie, il tendait vers lui les mains, avec transports.

\begin{flushright}
\emph{Romain Rolland}
\end{flushright}

\pagebreak
\thispagestyle{empty}
\movetooddpage

J'ai toujours vécu, parallèlement, deux vies, -- l'une, celle du
personnage que les combinaisons des éléments héréditaires m'ont fait
revêtir, dans un lieu de l'espace et une heure du temps, -- l'autre,
celle de l'Etre sans visage, sans nom, sans lieu, sans siècle, qui est
la substance même et le souffle de toute vie. Mais de ces deux
consciences, distinctes et conjuguées, -- l'une épidermique et fugace,
-- l'autre, durable et profonde, -- la première a, comme il est naturel,
recouvert la seconde, pendant la plus grande part de mon enfance, de ma
jeunesse, et même de ma vie active et passionnelle. Ce n'est que par
soudaines explosions que la conscience souterraine, réussissant à forer
l'écorce des jours, jaillit comme un jet brûlant de puits artésien, --
pour quelques secondes seulement, -- de nouveau disparue et sucée par
les lèvres de la terre. Jusqu'aux temps accomplis de la maturité, où les
coups répétés des blessures de la vie élargissant les fissures de
l'écorce, la poussée de l'âme intérieure fraie à l'Etre caché son
thalweg et son lit de fleuve dans la plaine.

Avant d'en arriver à cet état de communion directe, où je suis à
présent, avec la Vie universelle, j'ai vécu séparé d'elle et proche,
l'entendant cheminer avec moi, sous le rocher, -- et soudain, de loin en
loin, aux instants que je m'y attendais le moins, vivifié par ces
irruptions de flots artésiens, qui me frappaient à la face et qui me
terrassaient.

J'ai noté trois de ces jets de l'âme, trois de ces Eclairs, qui
remplirent mes veines du feu qui fait battre le c\oe ur de l'univers. La
trace de leur brûlure est restée aussi vive en mon vieux corps, que
l'épreuve a, depuis, roulé comme un galet, qu'à la minute lointaine où
elle s'imprimait dans la chair délicate et fiévreuse de l'adolescent.

Je ne livrerai ici que le récit du second de ces Eclairs~: les mots de
feu de Spinoza.

Entre seize et dix-huit ans.

Deux tragiques années. Insignifiantes, aux yeux qui n'en verraient que
l'étoffe, la vie familiale et scolaire d'un incertain adolescent. Mais
elles ont recélé les monstres dévorants du désespoir mortel. En ces
jours, non en d'autres, j'ai touché le fond du néant.

-- «~O aimable jeunesse~!~» me dit amèrement Spitteler en pensant à la
sienne\ldots{}

J'y reviendrai ailleurs\ldots{} Seule, tandis que je sombrais, la tempête de
Shakespeare, soulevant les couches profondes du morne océan, ramenait,
par remous, mon épave à la surface, pour la replonger \linebreak

\quebra

\noindent{}dans la nuit. Je
dirai le compagnon que me fut alors Hamlet, et le commentaire, agrippé à
chaque mot comme un lierre, que je lui ai consacré\ldots{}

Mais dans l'esprit s'opérait une métamorphose. Puissante et déchirante.
Je muais, de corps et d'âme, de la voix comme de la pensée. A seize ans,
mon intelligence était encore fermée aux idées abstraites. Je
traversais, en aveugle, la classe de philosophie, au lycée Saint-Louis,
avec Evelyn et Darlu. Devant ces mots sans visage, sans couleur, sans
odeur, que les mains ne pouvaient palper, que la bouche ne pouvait
mordre, qui se refusaient à la caresse comme à la blessure des sens, ces
mots-machines de la métaphysique et des mathématiques, instruments de
génie créés par le cerveau, je restais privé de souffle et hostile\ldots{}
«~\emph{Fuori Barbari~!\ldots{}}~». -- Or, moins d'un an après, en classe de
philo que je redoublais à Louis-le-Grand, pour me préparer à Normale,
j'étais devenu le premier~; et l'excellent Monsieur Charpentier, mon
maître grand et gros, jubilait en donnant à pleine voix lecture à son
troupeau de mes dissertations~: où d'ailleurs, traîtreusement, metteur
en scène de la pensée, je faisais dialoguer Malebranche avec son
chien\ldots{} La porte était ouverte et j'enjambais le \linebreak

\quebra

\noindent{}seuil du royaume de
l'Informe, -- sans doute en l'anthropomorphisant -- mais combien de
philosophes (et je dis des plus grands) ont été moins naïfs, ou plus
outrecuidants~!

Le cercle philosophique était assez étroit, dans la classe de Philo A de
Louis-le-Grand. Mais soigneusement pioché et retourné. Il restait
confiné entre les hautes haies du jardin de Descartes, Versailles de la
pensée. J'ai été substantifiquement nourri de la moelle Cartésienne,
pendant deux à trois années.

J'y ajoutais ce que j'allais grappiller dans l'enclos voisin (Philo B),
à la vigne de Burdeau, des fantasmagories Présocratiques. Quelques
graines tombées du bec de ces grands oiseaux, Ioniens et Trinacriens,
ont germé depuis, dans mon «~\emph{Empédocle d'Agrigente}~».

-- Mais le chemin naturel de l'esprit me conduisait à ceux qui, partis
du majestueux jardin muré de Descartes, y avaient, par une brèche,
ouvert des perspectives illimitées. Il me mena tout droit, d'instinct,
comme un chien, guidé par le flair de deux ou trois mots -- à Spinoza.

J'ai gardé précieusement l'édition, devenue rare, achetée sous les
galeries de l'Odéon, -- qui fut, en ces années, mon élixir de vie
éternelle~:

\emph{\OE uvres de Spinoza, traduites par Emile Saisset, avec une
introduction critique, nouvelle édition revue et augmentée}.
Charpentier, 1872, 3 volumes, in-12, cartonnés en vert.

Bien que ma pensée soit maintenant affranchie du strict rationalisme de
maître Benoît, et qu'elle en ait reconnu maints paralogismes, il me
reste sacré, à l'égal des Livres Saints pour un qui croit en eux~; et je
ne touche ces trois volumes qu'avec un pieux amour.

Je n'oublierai jamais que dans le cyclone de mon adolescence, j'ai
trouvé mon refuge au nid profond de l'\emph{Ethique}.

C'est quatre heures. L'hiver. Le jour tombe. Le jour terne d'un ciel
gris et glacé. Je suis assis devant ma table adossée au mur, près de la
fenêtre. Dehors, la rue Michelet, déserte, où s'engouffre la bise, et,
séparé par une grille, le funèbre jardin de l'Ecole de pharmacie, où les
rares visiteurs semblent prier devant les tombes des plantes.
Mais je ne vois rien du dehors. Je suis muré. Muré
dans la chambre close. Muré dans ma carapace hérissée contre le froid,
qui pénètre dans la pièce non chauffée et jusque sous le pardessus où se
recroqueville mon corps frileux. Muré dans la contemplation \linebreak

\quebra

\noindent{}du livre,
que tiennent mes doigts gourds. Autour de moi, je sens, morne halo, le
triste jour qui meurt, l'implacable nature, l'étau de la ville de
pierre, et celui de mes pensées.

L'éternel prisonnier, attaché dans sa geôle, traîne au pied le boulet du
souci, de la lutte pour la vie, l'obsession acharnée de l'examen, qui
empoisonne tant de jeunes existences, des échecs répétés, de la
nécessité de crisper toutes ses forces pour le combat, l'obligation
morale de vaincre, non seulement pour vivre, pour sauver sa vie, mais
pour sauver celle des siens, pour répondre à leur sacrifice absolu, qui
a misé tout leur sort sur une carte, sur mon sort. Malheureux enfant
débile, sur qui pèse une responsabilité trop lourde, qu'il n'a pas
demandée~! Elle l'oppresse~; et pourtant, elle lui est une armure~; en
écrasant ses épaules, elle l'oblige à se raidir. Sans elle, il
s'abandonnerait au Rêve incessant, qui bourdonne dans le fond de la
ruche fermée. Mais sous la chape qui le recouvre, sa frêle et nerveuse
énergie se concentre, se tend, angoissée, vers une lueur qui filtre par
l'étroit soupirail\ldots{}

\quebra

Elle filtre. Je la fixe dans la nuit de ma cave. Je la fixe entre les
barreaux noirs des lignes du livre vêtu de vert. Et sous la fixité
trouble de mon regard halluciné, voici que les barreaux s'écartent et
que surgit le soleil blanc de la \emph{Substance}. Métal en fusion, qui
remplit la coupe de mes yeux, il coule dans mon être qu'il consume~; et
mon être, comme une fonte, rejaillit en la cuve\ldots{}

Il a suffit d'une page, la première, -- de quatre Définitions, et de
quelques éclats de feu qui ont sauté, au choc des silex de
l'\emph{Ethique}.

Je ne me fais point d'illusion, et ne veux pas en faire aux autres. Je
ne prétends, ni que cette vertu de miracle soit inhérente à des mots
magiques, ni que j'y aie alors saisi la pensée vraie de Spinoza. De même
qu'en lisant le long premier volume d'\emph{Introduction}, honnête et
timorée, par Emile Saisset, je ne m'arrêtai pas aux arguments
effarouchés de ce spiritualiste et sautai allègrement pardessus son
garde-feu dans le braisier, dont son labeur avait pour objet de me
défendre -- (Naïfs contradicteurs~! C'est à eux que l'ont doit de
connaître et d'aimer les génies interdits~!) -- ainsi, dans le texte
même de Spinoza, je découvrais non lui, mais moi ignoré. Dans
l'inscription tracée au porche de l'\emph{Ethique}, dans ces Définitions
aux lettres flamboyantes, je déchiffrais, non ce qu'il avait dit, mais
ce que je voulais dire, les mots que ma propre pensée d'enfant, de sa
langue inarticulée, s'évertuait à épeler. On ne lit jamais un livre. On
se lit à travers les livres, soit pour se découvrir, soit pour se
contrôler. Et les plus objectifs sont les plus illusionnés. Le plus
grand livre n'est pas celui dont le communiqué s'imprimerait au cerveau,
ainsi que sur le rouleau de papier un message télégraphique, mais celui
dont le choc vital éveille d'autres vies, et, de l'une à l'autre,
propage son feu qui s'alimente des essences diverses et, devenu
incendie, de forêt en forêt bondit.

Je n'essaierai donc pas d'expliquer ici le sens libérateur de la vraie
pensée de Spinoza, mais celui que j'y ai trouvé, parce que depuis
l'enfance mon obscure passion, à tatons, le cherchait.

Et certes, ce n'est point le maître de l'ordre géométrique --
«~\emph{Ethica ordine geometrico demonstrata}~» -- ce n'est point le
rationaliste qui m'a conquis en Spinoza, -- quelque jouissance
esthétique que me procurent les jeux magnifiques de la raison~: -- c'est
le réaliste.

Qu'il est étrange que cet aspect de la grande figure soit
recouvert, jusqu'à devenir invisible, par le lourd verbalisme
intellectuel des philosophes de profession~! Comment, du premier regard,
ne saisissent-ils pas ce regard, cette voix, ivres du Réel~!

\begin{quote}
Il est absolument nécessaire de tirer toutes nos idées des choses
physiques, c'est à dire des êtres réels, en allant, suivant la série des
causes, d'un être réel à un autre être réel, sans passer aux choses
abstraites et universelles, ni pour en conclure rien de réel, ni pour
les conclure de quelque être réel~: car l'un et l'autre interrompent la
marche véritable de l'entendement.
\end{quote}

N'est-ce pas un principe de réalisme halluciné, qui gouverne en ces mots
le \emph{Traité de l'Entendement}, ajoutant aussitôt
après, avec l'imperturbable assurance du visionnaire~:

\emph{\ldots{} Mais il faut remarquer que par la série des causes et des
êtres réels, je n'entends point ici la série des choses particulières et
changeantes, mais seulement la série des choses fixes et
éternelles}.

«~\emph{Les choses fixes et éternelles}~» sont «~\emph{réelles}~». Elles
sont \emph{le plus réel}. Et tout ce qui est \emph{réel} est
\emph{individuel}.

«~\emph{Les choses fixes et éternelles}~» sont
«~\emph{particulières}~». Point d'abstractions. Des
Essences. Des Etres. Tout est \emph{être~}: -- et les \emph{Modes}
innombrables et finis~; et l'infinité des \emph{Attributs} infinis~; et
l'Etre des êtres, la Substance, «~\emph{L'Etre unique, infini, l'être
qui est tout l'être, et hors duquel il n'y a rien}~».

Vertige~!\ldots{} Vin de feu~!\ldots{} Ma prison s'ouvre. Voilà donc la réponse,
obscurément conçue dans la douleur et dans le désespoir, appelée par des
cris de passion aux ailes brisées, obstinément cherchée, voulue, dans
les meurtrissures et les larmes de sang, la voilà rayonnante, la réponse
à l'énigme du Sphinx, qui m'étreint depuis l'enfance, -- à l'antinomie
accablante entre l'immensité de mon être intérieur et le cachot de mon
individu, qui m'humilie et qui m'étouffe~!

«~\emph{Nature naturante}~» et «~\emph{nature naturée}~»\ldots{}

C'est la même.

«~\emph{Tout ce qui est, est en Dieu}.~»

Et moi aussi, je suis en Dieu~! De ma chambre glacée, où tombe la nuit
d'hiver, je m'évade au gouffre de la Substance, dans le soleil blanc de
l'Etre.

Horizons inouïs~! Mon rêve, même en ses vols les plus délirants, est
dépassé. Non seulement mon corps et mon esprit, mon univers, baignent
dans des mers sans rivages, l'Etendue, la Pensée, dont nulle caravelle
ne pourra faire le tour. Mais, dans l'insondable immensité, j'entends
bruire, à l'infini, d'autres mers, d'autres mers inconnues, des
Attributs innommables, inconcevables, à l'infini. Et tous sont contenus
en l'Océan de l'Etre. Entre son pouce et son petit doigt, ils tiennent à
l'aise. L'intuition de Spinoza ouvre les cieux fermés, -- de deux
siècles en avance, pionnière des \emph{conquistadores} de la science
moderne. Et si, dans ces Nouveaux Mondes, elle sait et nous dit que,
sous notre forme humaine, nous n'aborderons jamais, elle nous communique
l'ivresse de la certitude qu'ils existent, qu'ils sont là, près de
nous~: ce n'est pas seulement un fait de connaissance, mais le battement
de c\oe ur d'une coexistence. Enrichissement prodigieux de mon univers, il
n'y a qu'un instant étranglé dans la cage de ma maigre poitrine~! Et mon
c\oe ur ne souffre pas de son énormité. Les ailes étendues, planant sur
ces espaces, souffle à souffle, seul à seul, fixant le regard, sans
ciller de la Face omniprésente -- «~\emph{Facies totius
universi}~» -- je me sens soutenu par
l'infaillible main de la Libre Nécessité, qui émane du Dieu. Je ne
tomberai point. Car je suis sien. Ma chute serait la sienne\ldots{}

\emph{Si una pars materiae annihilaretur, simul tota Extensio
evanesceret\ldots{}}

Je ne puis tomber qu'en Lui. Je suis calme.

Tout est calme. Je jouis de ma plénitude et de mon harmonie\ldots{}

\emph{\ldots{} Possédant par une sorte de nécessité éternelle la connaissance
de moi-même et de Dieu et des choses, jamais je ne cesse d'être~; et la
véritable paix de l'âme, je la possède pour toujours.}

Mais ces dernières lignes de l'\emph{Ethique}, il ne faut pas les lire
-- et je ne les lisais pas -- avec les yeux froids de l'intelligence. Il
faut y apporter la passion de son c\oe ur et l'ardeur de ses sens. Il faut
participer au spasme de cette «~Béatitude~», ainsi que lui-même il la
nomme, notre Krishna d'Europe, et qui est «~\emph{un amour}~» et une volupté, -- la plus voluptueuse
des jouissances humaines~:

\emph{Aeternitatem, hoc est, infinitam existendi, sive, invita
latinitate, essendi fruitionem.}

Goûtez la saveur sensuelle de ce latin barbare~; «~\emph{essendi
fruitio}~»~!\ldots{} De mes yeux, de mes mains, de ma langue, de tous les
pores de ma pensée, je l'ai goûtée. J'ai étreint l'Etre.

O rire de Zarathustra~! Je n'ai pas attendu Nietzsche pour te connaître.
Tu résonnes ici, mais de quelles harmonies plus belles et plus pleines~!
Et comme elles sont proches de celles de l'\emph{Ode à la Joie~}!\ldots{}

\begin{quote}
La joie est une passion qui augmente ou favorise la puissance du
corps\ldots{} La joie est bonne\ldots{} La gaieté ne peut avoir d'excès, et elle
est toujours bonne\ldots{}
Le rire est un pur sentiment de joie, et il ne
peut avoir d'excès, et il est bon\ldots{} Plus nous avons de joie, et plus
nous avons de perfection\ldots{}

\ldots{} Jouir de la nourriture, des parfums, des couleurs, des beaux
vêtements, de la musique, des jeux, des spectacles, et de tous les
divertissements que chacun peut se donner, sans dommage pour
personne\ldots{}

\ldots{} User des choses de la vie et en jouir autant que possible\ldots{}
Se réunir aux autres et tâcher de les unir, -- car tout ce qui
tend à les unir est bon -- s'efforcer de partager sa joie avec les
autres\ldots{} \emph{-- s'unir, en pleine
connaissance, avec toute la nature\ldots{}}

\emph{Seid umschlungen, Millionen~!\ldots{}}
\end{quote}

Embrassons-nous, millions d'êtres~!

\begin{flushright}
Juillet 1924
\end{flushright}} 
\ParallelRText{\setcounter{section}{0} %\pagebreak
\thispagestyle{empty}

Estas páginas sobre Espinosa, que fazem parte de um capítulo de
\emph{Confissões inéditas}, intituladas ``A viagem interior'', foram
publicadas apenas em uma longínqua revista da Ásia, em língua bengali:
\emph{Probasi} (1926), por meu amigo, o professor Kalidas Nag. E sobre
isso quero contar um fato emocionante, que mostra mais uma vez o
parentesco dos espíritos do Oriente e do Ocidente.

Algumas semanas após a publicação, Kalidas Nag recebeu, de uma prisão da
Índia, uma carta censurada de um jovem preso político bengali. O
prisioneiro, que havia lido o relato extático do adolescente francês
vendo infiltrar"-se através das barras de sua cela o sol branco do Ser,
reconheceu"-se no jovem irmão da Europa. E, de seu calabouço desconhecido
da Ásia, estendia as mãos para ele, comovido.

\begin{flushright}
\emph{Romain Rolland}
\end{flushright}

\pagebreak
\thispagestyle{empty}
\movetooddpage

Eu sempre vivi duas vidas paralelas: uma, a do personagem que as
combinações dos elementos hereditários me fizeram trajar, em um lugar do
espaço e em uma hora do tempo; a outra, a do Ser sem rosto, sem nome,
sem lugar, sem século, que é a substância mesma e o sopro de toda vida.
Mas, dessas duas consciências distintas e conjugadas --- uma epidérmica e
fugaz, a outra durável e profunda ---, a primeira, como é natural,
encobriu a segunda durante a maior parte da minha infância, da minha
juventude e mesmo da minha vida ativa e passional. Foi apenas por
repentinas explosões que a consciência subterrânea, perfurando a couraça
dos dias, jorrou como um jato fervente de poço artesiano --- por alguns
segundos, somente --- e desapareceu de novo, sorvida pelos lábios da
terra. Até a chegada da maturidade, quando os golpes repetidos das
injúrias da vida alargam as fissuras da casca, o impulso da alma escava
para o Ser oculto seu talvegue e seu leito de rio na planície.

Antes de alcançar esse estado de comunhão direta com a Vida universal em
que estou agora, eu vivi separado dela e próximo, ouvindo"-a caminhar
comigo, sob o rochedo --- e subitamente, de\est\ longe em longe, nos instantes
em que eu menos esperava, era vivificado por essas irrupções de
torrentes artesianas, que me golpeavam a face e me derrubavam.

Eu notei três desses jatos da alma, três desses Clarões, que inundaram
minhas veias com o fogo que faz bater o coração do universo. A marca de
sua queimadura permaneceu tão viva em meu velho corpo --- que a prova
desde então rolou como um seixo --- quanto no minuto longínquo em que se
imprimiu na carne delicada e febril do adolescente.

Relatarei aqui apenas o segundo desses Clarões --- as palavras de fogo de
Espinosa.

Entre dezesseis e dezoito anos

Dois trágicos anos. Insignificantes aos olhos que vissem somente a
fachada, a vida familiar e escolar de um adolescente incerto. Mas eles
escondiam os monstros devoradores do desespero mortal. Nesses dias, não
em outros, eu toquei o fundo do nada.

--- ``Oh amável juventude!'', diz"-me amargamente Spitteler, pensando na sua\ldots{}

Voltarei a isto mais adiante\ldots{} Enquanto eu naufragava, apenas a
tempestade de Shakespeare, agitando as camadas profundas do morno
oceano, trazia, por arrebatamentos, minha carcaça à superfície, para
mergulhá"-la novamente na noite. Vou dizer o companheiro que Hamlet me
foi então, e o comentário que dediquei a ele, agarrado a cada palavra
como planta trepadeira\ldots{}

Mas no espírito operava"-se uma metamorfose. Potente e dilacerante. Eu
emudecia, de corpo e alma, da voz como do pensamento. Aos dezesseis
anos, minha inteligência era ainda fechada para as ideias abstratas. Eu
atravessava às cegas as aulas de filosofia no colégio Saint"-Louis, com
Evelyn e Darlu. Diante dessas palavras sem rosto, sem cor, sem odor, que
as mãos não podiam palpar, que a boca não podia morder, que se recusavam
tanto à carícia quanto às ofensas dos sentidos, diante dessas
palavras"-máquinas da metafísica e da matemática, instrumentos de engenho
criados pelo cérebro, eu ficava sufocado e hostil\ldots{} \emph{Fuori
Barbari!\ldots{}}. --- Ora, menos de um ano depois, no curso de filosofia que
eu repetia no Louis"-le"-Grand para me preparar para a Escola Normal,
tornei"-me o primeiro da classe; e o excelente Sr. Charpentier, meu
mestre grande e gordo, jubilava lendo as minhas dissertações a plenos
pulmões para o seu rebanho: nas quais, aliás, traiçoeiramente, encenador
do pensamento, eu fazia Malebranche dialogar com seu cachorro\ldots{} A porta
estava aberta e eu ultrapassava a fronteira do reino do Informe --- sem
dúvida antropomorfizando"-o --- mas quantos filósofos (e eu falo dos
maiores) foram menos ingênuos, ou mais presunçosos!

O círculo filosófico da turma ``Philo A'' do Louis"-le"-Grand era bem
pequeno. Mas cuidadosamente escolhido e examinado. Ficava confinado
entre as altas cercas"-vivas do jardim de Descartes, a Versailles do
pensamento. Fui substancialmente nutrido da medula cartesiana durante
dois ou três anos.

A isso eu acrescentava o que ia colher no terreno
vizinho (``Philo B''), no vinhedo de Burdeau, as fantasmagorias
pré"-socráticas. Algumas sementes caídas do bico desses grandes pássaros
jônicos e eleáticos germinaram depois, em meu ``Empédocles de
Agrigento''.

--- Mas o caminho natural do espírito me conduzia àqueles
que, partindo do majestoso jardim murado de Descartes, tinham aberto,
por uma brecha, perspectivas ilimitadas. Ele me levou diretamente, por
instinto, tal como um cão guiado pelo cheiro de duas ou três palavras, a
Espinosa.

Eu guardei preciosamente a edição --- hoje rara, comprada nas galerias do
Odeon --- que foi, nesses anos, meu elixir da vida eterna:

\emph{Obras de
Espinosa, traduzidas por Emile Saisset, com uma introdução crítica, nova
edição revista e ampliada}. Charpentier, 1872, 3 volumes in"-12,
encadernadas em verde.

Embora meu pensamento esteja agora livre do estrito racionalismo do
mestre Benoît, e tenha reconhecido nele muitos paralogismos, essa edição
continua sagrada para mim, como os Livros Sagrados para quem crê neles,
e eu não toco esses três volumes sem um amor piedoso.

Não esquecerei jamais que no ciclone de minha adolescência encontrei meu
refúgio no ninho profundo da \emph{Ética}.

São quatro horas. Inverno. O dia cai. O dia terno com um céu cinzento e
gelado. Estou sentado diante de minha mesa encostada na parede, perto da
janela. No exterior, a rua Michelet deserta, onde assovia o vento frio
do norte, e, separado por uma grade, o fúnebre jardim da Escola de
Farmácia, onde os raros visitantes parecem rezar diante das tumbas das
plantas\footnote{Desde então a vegetação cresceu. Na época, o jardim
  havia acabado de ser inaugurado e era pedregoso.}. Mas não vejo nada
lá fora. Estou enclausurado. Enclausurado no quarto fechado.
Enclausurado em minha carapaça eriçada contra o frio, que penetra no
cômodo sem aquecimentos e entra por baixo do sobretudo onde se encolhe
meu corpo friorento. Enclausurado na contemplação do livro que meus
dedos gélidos seguram. Ao meu redor, como uma aura mórbida, sinto o
triste dia que perece, a implacável natureza, a prisão da cidade de
pedra, e a dos meus pensamentos.

O eterno prisioneiro, cativo em seu
calabouço, arrasta o peso da preocupação, da luta pela vida, da obsessão
intransigente dos exames que envenena tantas jovens existências, dos
fracassos repetidos, da necessidade de retesar todas as forças para o
combate, não obstante o desgosto do combate, da obrigação moral de
vencer, não apenas para viver, para salvar a própria vida, mas para
salvar a vida dos seus, para corresponder ao sacrifício absoluto dos que
apostaram toda a sua sorte em uma só carta: a minha sorte. Frágil e
infeliz criança, sobre a qual pesa uma responsabilidade excessiva que
ela não pediu! Que a oprime, e, no entanto, serve"-lhe de armadura;
pesando sobre seus ombros, ela obriga a criança a endireitar"-se. Sem
ela, ter"-se"-ia abandonado ao Sonho incessante que zumbe do fundo da
colmeia fechada. Mas, sob a capa que a recobre, sua fugidia e nervosa
energia concentra"-se, estende"-se, angustiada, em direção a um luar que
se infiltra pela pequena claraboia\ldots{}

Ele se infiltra. Eu o fixo na noite do meu porão. Eu o fixo entre as
barras escuras das linhas do livro encapado de verde. E sob a fixidez
turva de
meu olhar alucinado, eis que as barras se afastam e surge o sol
branco da \emph{Substância}. Metal em fusão, que enche a taça dos meus
olhos, escorre em meu ser, consumindo"-o; e meu ser, como uma fonte,
jorra novamente na cuba\ldots{}

Bastou uma página, a primeira --- quatro Definições e algumas faíscas que
saltaram ao atrito das pedras de sílex da \emph{Ética}\footnote{Baruch
  de Espinosa, \emph{Ética} \versal{I}, Definições 3, 4, 5, 6 e a Explicação que
  se segue. Centelhas arrancadas das proposições 15 e 16 da parte \versal{I}, até
  o Escólio do Lema 7 da parte \versal{II}.}.

Não tenho ilusões e nem quero criá"-las nos outros. Eu não pretendo que
essa virtude de milagre seja inerente a palavras mágicas, nem acredito
que eu tenha apreendido ali o pensamento verdadeiro de Espinosa. Tanto
que, lendo o longo primeiro volume de \emph{Introdução}, honesto e
receoso, de Emile Saisset, eu não me detinha aos argumentos amedrontados
desse espiritualista, e saltava alegremente por sobre o parapeito no
braseiro de que ele procurava, com seu trabalho, defender"-me --- Ingênuos
opositores! É graças a eles que conhecemos e amamos os gênios proibidos!
--- assim, no próprio texto de Espinosa, eu descobria não a ele, mas a
mim ignorado. Na inscrição traçada no umbral da \emph{Ética}, nessas
Definições com letras flamejantes, eu decifrava não o que ele dissera,
mas o que eu queria dizer, as palavras que meu próprio pensamento de
criança, em sua língua inarticulada, empenhava"-se em soletrar. Nunca se
lê um livro. Lê"-se a si mesmo através dos livros, seja para
descobrir"-se, seja para controlar"-se. E os mais objetivos são os mais
iludidos. O maior livro não é aquele cujo comunicado se imprime no
cérebro tal como a mensagem telegráfica sobre um rolo de papel, mas
aquele cujo choque vital desperta outras vidas e, de uma a outra,
propaga seu fogo, que se alimenta das essências diversas e, tornando"-se
incêndio, de floresta em floresta se alastra.

Não tentarei, portanto, explicar aqui o sentido libertador do verdadeiro
pensamento de Espinosa, mas aquele que ali encontrei porque desde a
infância minha obscura paixão, tateando, o procurava.

E certamente não foi o mestre da ordem geométrica (\emph{Ethica ordine
geometrico demonstrata}), não foi o racionalista que me conquistou em
Espinosa (algum prazer estético que me oferecessem os jogos magníficos
da razão): foi o realista.

Como é estranho que esse aspecto da grande figura esteja recoberto, a
ponto de tornar"-se invisível, pelo pesado verbalismo intelectual dos\est\
filósofos de profissão! Como, ao primeiro vislumbre, eles não captam esse
olhar, essa voz, ébrios do Real!

\begin{quote}
Disso podemos ver ser"-nos antes de tudo necessário que sempre deduzamos
todas as nossas ideias das coisas físicas, ou seja, dos seres reais,
indo, quanto se pode fazer segundo a série das causas, de um ser real
para outro ser real, de modo a não passarmos a ideias abstratas e
universais, quer não deduzindo delas nada de real, quer não as
concluindo de coisas reais. Ambas as coisas, com efeito, interrompem o
verdadeiro progresso do intelecto.
\end{quote}

Não é um princípio de realismo alucinado que governa com essas palavras
o \emph{Tratado da Correção do Intelecto}\footnote{Id., \emph{Tratado da
  correção do intelecto}, trad. Marilena de Souza Chauí, in \emph{Os
    Pensadores} vol. \versal{XVII}. São Paulo: Abril Cultural, 1973, p.~73.},
acrescentando logo em seguida, com a imperturbável segurança do
visionário:

``Note"-se, porém, que por série das causas e dos seres reais
não entendo aqui a série das coisas singulares e móveis, mas apenas a
série das coisas fixas e eternas.''\footnote{Id., ``\emph{Per realitatem
  et perfectionem idem intelligo}'' in \emph{Ética}, \versal{II}, Definição 6.
  {[}``Por realidade e perfeição entendo o mesmo'', trad. Marilena Chauí
  e Grupo de Estudos Espinosanos. São Paulo: Edusp, 2015, p.~127.{]}}

``As coisas fixas e eternas'' são ``reais''. Elas são \emph{o mais real}. E
tudo o que é \emph{real} é \emph{individual}.

``As coisas fixas e eternas'' são
``singulares''.\footnote{Id. \emph{Tratado da correção do intelecto},
  op.~cit. p.~73--74.} Sem mais abstrações. Essências. Seres. Tudo é \emph{ser}:
e os \emph{Modos} inumeráveis e finitos; e a infinidade dos \emph{Atributos}
infinitos; e o Ser dos seres, a Substância, ``este ser é único,
infinito, quer dizer, todo o ser, e fora dele não há ser
algum''.\footnote{Ibid., p.~68.}

Vertigem!\ldots{} Vinho de fogo!\ldots{} Minha prisão se abre. Eis então a
resposta, obscuramente concebida na dor e no desespero, invocada por
gritos de paixão com as asas quebradas, obstinadamente buscada,
desejada, ei"-la flamejante, a resposta ao enigma da Esfinge, que me
abraça desde a infância --- à antinomia ultrajante entre a imensidão de
meu ser interior e a masmorra de meu indivíduo que me avilta e me
sufoca!

``Natureza naturante'' e ``Natureza naturada''\footnote{Id.,
  \emph{Ética} \versal{I}, 29, Escólio, op.~cit., p.~97.}\ldots{}

São a mesma.

``Tudo que é, é em Deus''.\footnote{Ibid., \versal{I}, 15, p.~67.}

E eu também, eu sou em Deus! De meu quarto gelado, onde cai a noite de
inverno, evado"-me no abismo da Substância, no sol branco do Ser.

Horizontes inauditos! Meu sonho, mesmo em seus voos mais delirantes, foi
ultrapassado. Não somente meu corpo e meu espírito, também meu universo
banham"-se em mares sem limites, a Extensão, o Pensamento, cuja vastidão
nenhuma caravela poderá contornar. Mas, na insondável imensidão, escuto
murmurar, ao infinito, outros mares, outros mares desconhecidos,
Atributos\est\ inomináveis, inconcebíveis, ao infinito. E todos estão
contidos no Oceano do Ser. Entre seu polegar e seu dedo mínimo eles
acomodam"-se à vontade. A intuição de Espinosa abre os céus cerrados ---
com dois séculos de antecedência, pioneira dos \emph{conquistadores} da
ciência moderna. E se tal intuição sabe e nos diz que, sob nossa forma
humana, nós não aportaremos jamais nesses Novos Mundos, também nos
comunica a embriaguez da certeza de que eles existem, de que eles estão
aqui, perto de nós: e não é somente um fato do conhecimento, mas as
batidas do coração de uma coexistência. Enriquecimento prodigioso do meu
universo, há apenas um instante sufocado na gaiola de meu estreito
peito! E meu coração não sofre de sua enormidade. Com as asas
estendidas, planando nesses espaços, de fôlego em fôlego, solidão em
solidão, fitando, sem piscar, o olhar da Face onipresente ---
\emph{Facies totius universi}\footnote{Id., Carta 64 a Schuller. {[}``O
  aspecto do universo inteiro''{]} {[}edição francesa disponível:
  Spinoza, \emph{Correspondance}, org. Max Rovere, Paris, \versal{GF} Flammarion,
  2010{]}.} --- sinto"-me sustentado pela infalível mão da Livre
Necessidade, que emana do Deus. Eu não cairei. Porque sou dele. Minha
queda seria a sua\ldots{}

\emph{Si una pars materiae annihilaretur, simul
tota Extensio evanesceret\ldots{}}\footnote{Id., Carta 4 a Oldenburg {[}``Se
  uma única parte da matéria se aniquilasse, imediatamente toda Extensão
  evanesceria\ldots{}''{]} {[}edição francesa disponível: Spinoza,
  \emph{Correspondance}, org. Max Rovere, Paris, \versal{GF} Flammarion, 2010{]}.}

Eu não posso cair a não ser Nele. Estou calmo.

Tudo está calmo. Gozo de minha plenitude e de minha harmonia\ldots{}

``Possuindo, por uma espécie de necessidade eterna, o conhecimento de
mim mesmo e de Deus e das coisas, eu nunca deixo de ser; e a verdadeira
paz da alma, eu a possuo para sempre.''\footnote{Romain Rolland retoma
  em primeira pessoa trechos do Escólio da Proposição 42 (\emph{Ética},
  \versal{V}). {[}\versal{N.~T.}{]}}

Mas essas últimas linhas da \emph{Ética} não devem ser lidas --- e eu não
as lia --- com os olhos frios da inteligência. É preciso trazer"-lhes a
paixão do coração e o ardor dos sentidos. É preciso participar do
espasmo dessa ``Beatitude'', como ele mesmo a nomeia, nosso Krishna da
Europa, e que é ``um amor''\footnote{``\emph{O amor divino ou a
  beatitude\ldots{}}''} e uma volúpia --- o mais voluptuoso dos gozos humanos:

\emph{Aeternitatem, hoc est, infinitam existendi, sive, invita
latinitate, essendi fruitionem.}\footnote{B. Espinosa, Carta 12 a L.
  Meyer, trecho {[}5{]} {[}Trecho completo: ``Donde nasce a diferença
  entre eternidade e duração. Pela duração, nós podemos com efeito
  explicar somente a existência dos modos. E a da substância,
  \emph{pela eternidade, isto é, pela fruição infinita do existir (a
  despeito do latim) do ser}.''{]}. No trecho citado por Romain Rolland,
  e aqui em itálico, Espinosa escreve ``a despeito do latim'' porque a
  forma como ele utiliza o verbo \emph{ser} no final é um barbarismo que
  se distancia do latim clássico. {[}\versal{N.~T.}{]}}

Deguste o sabor sensual desse latim bárbaro, \emph{essendi fruitio}!\ldots{}
Com meus olhos, com minhas mãos, com minha língua, com todos os poros do
meu pensamento, eu o degustei. Eu alcancei o Ser.

Oh riso de Zaratustra! Eu não esperei Nietzsche para conhecê"-lo. Você
ressoa aqui, mas com que harmonias mais belas e mais plenas! E como elas
são próximas da \emph{Ode à alegria}!\ldots{}

\pagebreak

\begin{quote}
A alegria é um afeto pelo qual a potência de agir do corpo é
aumentada\ldots{} A alegria é diretamente boa\ldots{} A Hilaridade não pode ter
excesso, sendo sempre boa\ldots{} O riso, como o gracejo, é mera Alegria, e
por isso, contanto que não seja excessivo, é bom por si\ldots{} Quanto maior
é a Alegria com que somos afetados, tanto maior é a perfeição a que
passamos\ldots{}\footnote{Id., \emph{Ética}, \versal{IV}, 41, 42 e 45 (Escólio do
  Corolário 2), op.~cit., p.~443,~447,~449.}

\ldots{}Gozar moderadamente de comida e bebida agradáveis, assim como cada
um pode usar, sem qualquer dano a outrem, dos perfumes, da amenidade dos
bosques, do ornamento, da música, dos jogos esportivos, do teatro e de
outras coisas desse tipo''\ldots{}\footnote{Id., \emph{Ética}, \versal{IV}, 45
  (Escólio do Corolário 2), op.~cit., p.~449.}

\ldots{}Servir"-se das coisas da vida e usufruí"-las tanto quanto possível\ldots{}
Reunir"-se aos outros e tratar de os unir --- pois tudo o que tende a
uni"-los é bom --- esforçar"-se por partilhar sua alegria com os outros\ldots{}\footnote{Id.,
  \emph{Ética}, \versal{IV}, 40, op.~cit., p.~443.}
--- unir"-se, em pleno conhecimento, com toda a natureza\ldots{}\footnote{Id.,
  \emph{Tratado da correção do intelecto}, op.~cit., p.~53.}

\emph{Seid umschlungen, Millionem!\ldots{}}
\end{quote}

Abracemo"-nos, milhões de seres!

\begin{flushright}
\emph{Julho de 1924}
\end{flushright}} %

\renewcommand{\ParallelAtEnd}{\noindent{}\vspace{1cm}\Large{Notas}}
\end{Parallel} 

\end{document}
