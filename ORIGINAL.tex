\pagebreak
\thispagestyle{empty}

Estas páginas sobre Espinosa, que fazem parte de um capítulo de
\emph{Confissões inéditas}, intituladas ``A viagem interior'', foram
publicadas apenas em uma longínqua revista da Ásia, em língua bengali:
\emph{Probasi} (1926), por meu amigo, o professor Kalidas Nag. E sobre
isso quero contar um fato emocionante, que mostra mais uma vez o
parentesco dos espíritos do Oriente e do Ocidente.

Algumas semanas após a publicação, Kalidas Nag recebeu, de uma prisão da
Índia, uma carta censurada de um jovem preso político bengali. O
prisioneiro, que havia lido o relato extático do adolescente francês
vendo infiltrar"-se através das barras de sua cela o sol branco do Ser,
reconheceu"-se no jovem irmão da Europa. E, de seu calabouço desconhecido
da Ásia, estendia as mãos para ele, comovido.

\begin{flushright}
\emph{Romain Rolland}
\end{flushright}

\pagebreak
\thispagestyle{empty}
\movetooddpage

Eu sempre vivi duas vidas paralelas: uma, a do personagem que as
combinações dos elementos hereditários me fizeram trajar, em um lugar do
espaço e em uma hora do tempo; a outra, a do Ser sem rosto, sem nome,
sem lugar, sem século, que é a substância mesma e o sopro de toda vida.
Mas, dessas duas consciências distintas e conjugadas -- uma epidérmica e
fugaz, a outra durável e profunda --, a primeira, como é natural,
encobriu a segunda durante a maior parte da minha infância, da minha
juventude e mesmo da minha vida ativa e passional. Foi apenas por
repentinas explosões que a consciência subterrânea, perfurando a couraça
dos dias, jorrou como um jato fervente de poço artesiano -- por alguns
segundos, somente -- e desapareceu de novo, sorvida pelos lábios da
terra. Até a chegada da maturidade, quando os golpes repetidos das
injúrias da vida alargam as fissuras da casca, o impulso da alma escava
para o Ser oculto seu talvegue e seu leito de rio na planície.

Antes de alcançar esse estado de comunhão direta com a Vida universal em
que estou agora, eu vivi separado dela e próximo, ouvindo"-a caminhar
comigo, sob o rochedo -- e subitamente, de longe em longe, nos instantes
em que eu menos esperava, era vivificado por essas irrupções de
torrentes artesianas, que me golpeavam a face e me derrubavam.

Eu notei três desses jatos da alma, três desses Clarões, que inundaram
minhas veias com o fogo que faz bater o coração do universo. A marca de
sua queimadura permaneceu tão viva em meu velho corpo -- que a prova
desde então rolou como um seixo -- quanto no minuto longínquo em que se
imprimiu na carne delicada e febril do adolescente.

Relatarei aqui apenas o segundo desses Clarões -- as palavras de fogo de
Espinosa.

Entre dezesseis e dezoito anos

Dois trágicos anos. Insignificantes aos olhos que vissem somente a
fachada, a vida familiar e escolar de um adolescente incerto. Mas eles
escondiam os monstros devoradores do desespero mortal. Nesses dias, não
em outros, eu toquei o fundo do nada. -- ``Oh amável juventude!'',
diz"-me amargamente Spitteler, pensando na sua\ldots{}

Voltarei a isto mais adiante\ldots{} Enquanto eu naufragava, apenas a
tempestade de Shakespeare, agitando as camadas profundas do morno
oceano, trazia, por arrebatamentos, minha carcaça à superfície, para
mergulhá"-la novamente na noite. Vou dizer o companheiro que Hamlet me
foi então, e o comentário que dediquei a ele, agarrado a cada palavra
como planta trepadeira\ldots{}

Mas no espírito operava"-se uma metamorfose. Potente e dilacerante. Eu
emudecia, de corpo e alma, da voz como do pensamento. Aos dezesseis
anos, minha inteligência era ainda fechada para as ideias abstratas. Eu
atravessava às cegas as aulas de filosofia no colégio Saint"-Louis, com
Evelyn e Darlu. Diante dessas palavras sem rosto, sem cor, sem odor, que
as mãos não podiam palpar, que a boca não podia morder, que se recusavam
tanto à carícia quanto às ofensas dos sentidos, diante dessas
palavras"-máquinas da metafísica e da matemática, instrumentos de engenho
criados pelo cérebro, eu ficava sufocado e hostil\ldots{} \emph{Fuori
Barbari!\ldots{}}. -- Ora, menos de um ano depois, no curso de filosofia que
eu repetia no Louis"-le"-Grand para me preparar para a Escola Normal,
tornei"-me o primeiro da classe; e o excelente Sr. Charpentier, meu
mestre grande e gordo, jubilava lendo as minhas dissertações a plenos
pulmões para o seu rebanho: nas quais, aliás, traiçoeiramente, encenador
do pensamento, eu fazia Malebranche dialogar com seu cachorro\ldots{} A porta
estava aberta e eu ultrapassava a fronteira do reino do Informe -- sem
dúvida antropomorfizando"-o --, mas quantos filósofos (e eu falo dos
maiores) foram menos ingênuos, ou mais presunçosos!

O círculo filosófico da turma ``Philo A'' do Louis"-le"-Grand era bem
pequeno. Mas cuidadosamente escolhido e examinado. Ficava confinado
entre as altas cercas"-vivas do jardim de Descartes, a Versailles do
pensamento. Fui substancialmente nutrido da medula cartesiana durante
dois ou três anos.

A isso eu acrescentava o que ia colher no terreno
vizinho (``Philo B''), no vinhedo de Burdeau, as fantasmagorias
pré"-socráticas. Algumas sementes caídas do bico desses grandes pássaros
jônicos e eleáticos germinaram depois, em meu ``Empédocles de
Agrigento''.

-- Mas o caminho natural do espírito me conduzia àqueles
que, partindo do majestoso jardim murado de Descartes, tinham aberto,
por uma brecha, perspectivas ilimitadas. Ele me levou diretamente, por
instinto, tal como um cão guiado pelo cheiro de duas ou três palavras, a
Espinosa.

Eu guardei preciosamente a edição -- hoje rara, comprada nas galerias do
Odeon -- que foi, nesses anos, meu elixir da vida eterna:

\emph{Obras de
Espinosa, traduzidas por Emile Saisset, com uma introdução crítica, nova
edição revista e ampliada}. Charpentier, 1872, 3 volumes in"-12,
encadernadas em verde.

Embora meu pensamento esteja agora livre do estrito racionalismo do
mestre Benoît, e tenha reconhecido nele muitos paralogismos, essa edição
continua sagrada para mim, como os Livros Sagrados para quem crê neles,
e eu não toco esses três volumes sem um amor piedoso.

Não esquecerei jamais que no ciclone de minha adolescência encontrei meu
refúgio no ninho profundo da \emph{Ética}.

São quatro horas. Inverno. O dia cai. O dia terno com um céu cinzento e
gelado. Estou sentado diante de minha mesa encostada na parede, perto da
janela. No exterior, a rua Michelet deserta, onde assovia o vento frio
do norte, e, separado por uma grade, o fúnebre jardim da Escola de
Farmácia, onde os raros visitantes parecem rezar diante das tumbas das
plantas\footnote{Desde então a vegetação cresceu. Na época, o jardim
  havia acabado de ser inaugurado e era pedregoso.}. Mas não vejo nada
lá fora. Estou enclausurado. Enclausurado no quarto fechado.
Enclausurado em minha carapaça eriçada contra o frio, que penetra no
cômodo sem aquecimentos e entra por baixo do sobretudo onde se encolhe
meu corpo friorento. Enclausurado na contemplação do livro que meus
dedos gélidos seguram. Ao meu redor, como uma aura mórbida, sinto o
triste dia que perece, a implacável natureza, a prisão da cidade de
pedra, e a dos meus pensamentos.

O eterno prisioneiro, cativo em seu
calabouço, arrasta o peso da preocupação, da luta pela vida, da obsessão
intransigente dos exames que envenena tantas jovens existências, dos
fracassos repetidos, da necessidade de retesar todas as forças para o
combate, não obstante o desgosto do combate, da obrigação moral de
vencer, não apenas para viver, para salvar a própria vida, mas para
salvar a vida dos seus, para corresponder ao sacrifício absoluto dos que
apostaram toda a sua sorte em uma só carta: a minha sorte. Frágil e
infeliz criança, sobre a qual pesa uma responsabilidade excessiva que
ela não pediu! Que a oprime, e, no entanto, serve"-lhe de armadura;
pesando sobre seus ombros, ela obriga a criança a endireitar"-se. Sem
ela, ter"-se"-ia abandonado ao Sonho incessante que zumbe do fundo da
colmeia fechada. Mas, sob a capa que a recobre, sua fugidia e nervosa
energia concentra"-se, estende"-se, angustiada, em direção a um luar que
se infiltra pela pequena claraboia\ldots{}

Ele se infiltra. Eu o fixo na noite do meu porão. Eu o fixo entre as
barras escuras das linhas do livro encapado de verde. E sob a fixidez
turva de meu olhar alucinado, eis que as barras se afastam e surge o sol
branco da \emph{Substância}. Metal em fusão, que enche a taça dos meus
olhos, escorre em meu ser, consumindo"-o; e meu ser, como uma fonte,
jorra novamente na cuba\ldots{}

Bastou uma página, a primeira -- quatro Definições e algumas faíscas que
saltaram ao atrito das pedras de sílex da \emph{Ética}\footnote{Baruch
  de Espinosa, {\slsc{Ética}} \scalebox{0.8}{I}, Definições 3, 4, 5, 6 e a Explicação que
  se segue. Centelhas arrancadas das proposições 15 e 16 da parte \scalebox{0.8}{I}, até
  o Escólio do Lema 7 da parte \scalebox{0.8}{II}.}.

Não tenho ilusões e nem quero criá"-las nos outros. Eu não pretendo que
essa virtude de milagre seja inerente a palavras mágicas, nem acredito
que eu tenha apreendido ali o pensamento verdadeiro de Espinosa. Tanto
que, lendo o longo primeiro volume de \emph{Introdução}, honesto e
receoso, de Emile Saisset, eu não me detinha aos argumentos amedrontados
desse espiritualista, e saltava alegremente por sobre o parapeito no
braseiro de que ele procurava, com seu trabalho, defender"-me -- Ingênuos
opositores! É graças a eles que conhecemos e amamos os gênios proibidos!
--; assim, no próprio texto de Espinosa, eu descobria não a ele, mas a
mim ignorado. Na inscrição traçada no umbral da \emph{Ética}, nessas
Definições com letras flamejantes, eu decifrava não o que ele dissera,
mas o que eu queria dizer, as palavras que meu próprio pensamento de
criança, em sua língua inarticulada, empenhava"-se em soletrar. Nunca se
lê um livro. Lê"-se a si mesmo através dos livros, seja para
descobrir"-se, seja para controlar"-se. E os mais objetivos são os mais
iludidos. O maior livro não é aquele cujo comunicado se imprime no
cérebro tal como a mensagem telegráfica sobre um rolo de papel, mas
aquele cujo choque vital desperta outras vidas e, de uma a outra,
propaga seu fogo, que se alimenta das essências diversas e, tornando"-se
incêndio, de floresta em floresta se alastra.

Não tentarei, portanto, explicar aqui o sentido libertador do verdadeiro
pensamento de Espinosa, mas aquele que ali encontrei porque desde a
infância minha obscura paixão, tateando, o procurava.

E certamente não foi o mestre da ordem geométrica (\emph{Ethica ordine
geometrico demonstrata}), não foi o racionalista que me conquistou em
Espinosa (algum prazer estético que me oferecessem os jogos magníficos
da razão): foi o realista.

Como é estranho que esse aspecto da grande figura esteja recoberto, a
ponto de tornar"-se invisível, pelo pesado verbalismo intelectual dos
filósofos de profissão! Como, ao primeiro olhar, eles não captam esse
olhar, essa voz, ébrios do Real!

\begin{quote}
Disso podemos ver ser"-nos antes de tudo necessário que sempre deduzamos
todas as nossas ideias das coisas físicas, ou seja, dos seres reais,
indo, quanto se pode fazer segundo a série das causas, de um ser real
para outro ser real, de modo a não passarmos a ideias abstratas e
universais, quer não deduzindo delas nada de real, quer não as
concluindo de coisas reais. Ambas as coisas, com efeito, interrompem o
verdadeiro progresso do intelecto.
\end{quote}

Não é um princípio de realismo alucinado que governa com essas palavras
o \emph{Tratado da Correção do Intelecto}\footnote{Id., {\slsc{Tratado da
  correção do intelecto}}, trad. Marilena de Souza Chauí, in {\slsc{Os
    Pensadores}} vol. \scalebox{0.8}{XVII}. São Paulo: Abril Cultural, 1973, p. 73.},
acrescentando logo em seguida, com a imperturbável segurança do
visionário:

``Note"-se, porém, que por série das causas e dos seres reais
não entendo aqui a série das coisas singulares e móveis, mas apenas a
série das coisas fixas e eternas.''\footnote{Id., ``{\slsc{Per realitatem
  et perfectionem idem intelligo}}'' in {\slsc{Ética}}, \scalebox{0.8}{II}, Definição 6.
  {[}``Por realidade e perfeição entendo o mesmo'', trad. Marilena Chauí
  e Grupo de Estudos Espinosanos. São Paulo: Edusp, 2015, p. 127.{]}}

``As coisas fixas e eternas'' são ``reais''. Elas são o mais real. E
tudo o que é real é individual.

``As coisas fixas e eternas'' são
``singulares''.\footnote{Id. {\slsc{Tratado da correção do intelecto}},
  op. cit. p. 73-74.} Sem mais abstrações. Essências. Seres. Tudo é ser:
e os Modos inumeráveis e finitos; e a infinidade dos Atributos
infinitos; e o Ser dos seres, a Substância, ``este ser é único,
infinito, quer dizer, todo o ser, e fora dele não há ser
algum''.\footnote{Ibid., p. 68.}

Vertigem!\ldots{} Vinho de fogo!\ldots{} Minha prisão se abre. Eis então a
resposta, obscuramente concebida na dor e no desespero, invocada por
gritos de paixão com as asas quebradas, obstinadamente buscada,
desejada, ei"-la flamejante, a resposta ao enigma da Esfinge, que me
abraça desde a infância -- à antinomia ultrajante entre a imensidão de
meu ser interior e a masmorra de meu indivíduo que me avilta e me
sufoca!

``Natureza naturante'' e ``Natureza naturada''\footnote{Id.,
  {\slsc{Ética}} \scalebox{0.8}{I}, 29, Escólio, op.~cit., p 97.}\ldots{}

São a mesma.

``Tudo que é, é em Deus''.\footnote{Ibid., \scalebox{0.8}{I}, 15, p. 67.}

E eu também, eu sou em Deus! De meu quarto gelado, onde cai a noite de
inverno, evado"-me no abismo da Substância, no sol branco do Ser.

Horizontes inauditos! Meu sonho, mesmo em seus voos mais delirantes, foi
ultrapassado. Não somente meu corpo e meu espírito, também meu universo
banham"-se em mares sem limites, a Extensão, o Pensamento, cuja vastidão
nenhuma caravela poderá contornar. Mas, na insondável imensidão, escuto
murmurar, ao infinito, outros mares, outros mares desconhecidos,
Atributos inomináveis, inconcebíveis, ao infinito. E todos estão
contidos no Oceano do Ser. Entre seu polegar e seu dedo mínimo eles
acomodam"-se à vontade. A intuição de Espinosa abre os céus cerrados --
com dois séculos de antecedência, pioneira dos \emph{conquistadores} da
ciência moderna. E se tal intuição sabe e nos diz que, sob nossa forma
humana, nós não aportaremos jamais nesses Novos Mundos, também nos
comunica a embriaguez da certeza de que eles existem, de que eles estão
aqui, perto de nós: e não é somente um fato do conhecimento, mas as
batidas do coração de uma coexistência. Enriquecimento prodigioso do meu
universo, há apenas um instante sufocado na gaiola de meu estreito
peito! E meu coração não sofre de sua enormidade. Com as asas
estendidas, planando nesses espaços, de fôlego em fôlego, solidão em
solidão, fitando, sem piscar, o olhar da Face onipresente --
\emph{Facies totius universi}\footnote{Id., Carta 64 a Schuller. {[}``O
  aspecto do universo inteiro''{]} {[}edição francesa disponível:
  Spinoza, {\slsc{Correspondance}}, org. Max Rovere, Paris, \scalebox{0.8}{GF} Flammarion,
  2010{]}.} -- sinto"-me sustentado pela infalível mão da Livre
Necessidade, que emana do Deus. Eu não cairei. Porque sou dele. Minha
queda seria a sua\ldots{} \emph{Si una pars materiae annihilaretur, simul
tota Extensio evanesceret\ldots{}}\footnote{Id., Carta 4 a Oldenburg {[}``Se
  uma única parte da matéria se aniquilasse, imediatamente toda Extensão
  evanesceria\ldots{}''{]} {[}edição francesa disponível: Spinoza,
  {\slsc{Correspondance}}, org. Max Rovere, Paris, \scalebox{0.8}{GF} Flammarion, 2010{]}.}

Eu não posso cair a não ser Nele. Estou calmo.

Tudo está calmo. Gozo de minha plenitude e de minha harmonia\ldots{}

``Possuindo, por uma espécie de necessidade eterna, o conhecimento de
mim mesmo e de Deus e das coisas, eu nunca deixo de ser; e a verdadeira
paz da alma, eu a possuo para sempre.''\footnote{Romain Rolland retoma
  em primeira pessoa trechos do Escólio da Proposição 42 ({\slsc{Ética}},
  \scalebox{0.8}{V}). (\scalebox{0.8}{N.~T.})}

Mas essas últimas linhas da \emph{Ética} não devem ser lidas -- e eu não
as lia -- com os olhos frios da inteligência. É preciso trazer"-lhes a
paixão do coração e o ardor dos sentidos. É preciso participar do
espasmo dessa ``Beatitude'', como ele mesmo a nomeia, nosso Krishna da
Europa, e que é ``um amor''\footnote{``{\slsc{O amor divino ou a
  beatitude\ldots{}}}''} e uma volúpia -- o mais voluptuoso dos gozos humanos:

\emph{Aeternitatem, hoc est, infinitam existendi, sive, invita
latinitate, essendi fruitionem.}\footnote{B. Espinosa, Carta 12 a L.
  Meyer, trecho {[}5{]} {[}Trecho completo: ``Donde nasce a diferença
  entre eternidade e duração. Pela duração, nós podemos com efeito
  explicar somente a existência dos modos. E a da substância,
  \textbf{pela eternidade, isto é, pela fruição infinita do existir (a
  despeito do latim) do ser}.''{]}. No trecho citado por Romain Rolland,
  e aqui em negrito, Espinosa escreve ``a despeito do latim'' porque a
  forma como ele utiliza o verbo {\slsc{ser}} no final é um barbarismo que
  se distancia do latim clássico. (\scalebox{0.8}{N.~T.})}

Deguste o sabor sensual desse latim bárbaro, \emph{essendi fruitio}!\ldots{}
Com meus olhos, com minhas mãos, com minha língua, com todos os poros do
meu pensamento, eu o degustei. Eu alcancei o Ser.

Oh riso de Zaratustra! Eu não esperei Nietzsche para conhecê"-lo. Você
ressoa aqui, mas com que harmonias mais belas e mais plenas! E como elas
são próximas da \emph{Ode à alegria}!\ldots{}

\begin{quote}
A alegria é um afeto pelo qual a potência de agir do corpo é
aumentada\ldots{} a alegria é diretamente boa\ldots{} A Hilaridade não pode ter
excesso, sendo sempre boa\ldots{} O riso, como o gracejo, é mera Alegria, e
por isso, contanto que não seja excessivo, é bom por si\ldots{} Quanto maior
é a Alegria com que somos afetados, tanto maior é a perfeição a que
passamos\ldots{}\footnote{Id., {\slsc{Ética}}, \scalebox{0.8}{IV}, 41, 42 e 45 (Escólio do
  Corolário 2), op. cit., p. 443, 447, 449.}

\ldots{}gozar moderadamente de comida e bebida agradáveis, assim como cada
um pode usar, sem qualquer dano a outrem, dos perfumes, da amenidade dos
bosques, do ornamento, da música, dos jogos esportivos, do teatro e de
outras coisas desse tipo''\ldots{}\footnote{Id., {\slsc{Ética}}, \scalebox{0.8}{IV}, 45
  (Escólio do Corolário 2), op. cit., p. 449.}

\ldots{}Servir"-se das coisas da vida e usufruí"-las tanto quanto possível\ldots{}
Reunir"-se aos outros e tratar de os unir -- pois tudo o que tende a
uni"-los é bom -- esforçar"-se por partilhar sua alegria com os outros\ldots{}\footnote{Id.,
  {\slsc{Ética}}, \scalebox{0.8}{IV}, 40, op. cit., 443.}
-- unir"-se, em pleno conhecimento, com toda a natureza\ldots{}\footnote{Id.,
  {\slsc{Tratado da correção do intelecto}}, op. cit. p. 53.}

\emph{Seid umschlungen, Millionem!\ldots{}}
\end{quote}

Abracemo"-nos, milhões de seres!

\begin{flushright}
\emph{Julho de 1924}
\end{flushright}